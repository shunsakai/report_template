% プリアンブル
\documentclass[a4j]{jarticle}
\usepackage{listings}
\usepackage[dvipdfmx]{graphicx}
\usepackage{amsmath}
\usepackage{bm}
\usepackage{here}
\usepackage{url}
\usepackage{verbatim}
\usepackage{ascmac}
\usepackage{graphicx}
\usepackage{multirow}
\usepackage{xcolor}
\usepackage{caption}
\DeclareCaptionFont{white}{\color{white}}
\DeclareCaptionFormat{listing}{
  \parbox{\textwidth}{\colorbox{gray}{\parbox{\textwidth}{#1#2#3}}\vskip-4pt}
}
\captionsetup[lstlisting]{format=listing,labelfont=white,textfont=white}
\lstset{frame=lrb,xleftmargin=\fboxsep,xrightmargin=-\fboxsep}
\belowcaptionskip=-10pt

% タイトル
\title{
  \vspace{-25mm}
  2016年度 実験\\レポート
}
\author{学生番号:\\氏名:}
\date{提出日}

% プログラム
\renewcommand{\lstlistingname}{Programs}
\lstset{
  language={C},
  basicstyle={\small\ttfamily},
  identifierstyle={\small},
  commentstyle={\small},
  keywordstyle={\small\bfseries},
  ndkeywordstyle={\small},
  stringstyle={\small},
  frame={tb},
  breaklines=true,
  columns=[l]{fullflexible},
  numbers=left,
  xrightmargin=0zw,
  xleftmargin=3zw,
  numberstyle={\scriptsize},
  stepnumber=1,
  numbersep=1zw,
  lineskip=-0.5ex,
  showstringspaces=false
}

\begin{document}
\setcounter{page}{1}

\maketitle

% abstract in Japanese
\begin{abstract}
  \include{src/abstract.tex}
\end{abstract}

\tableofcontents % 目次
\listoffigures % 表目次
\listoftables % 図目次

% 目的
%\input{src/1.tex}

% 背景・原理
%\input{src/2.tex}

% 実験内容
%\input{src/3.tex}

% 結果
%\input{src/4.tex}

% 考察
%\input{src/5.tex}

% 感想
%\include{src/impression.tex}

% 参考文献
%\begin{thebibliography}[90]
  % \bibitem{参照用名称}
  % 著者名:
  % \newblock 文献名,
  % \newblock 書誌情報,出版年
  \mbox{}
  \bibitem{bib01}
    ほげ井:
    \newblock hoge理論のHCI 分野への応用、
    \newblock hoge学会論文誌,Vol.31,No.3,pp.194-201,2009.
  \bibitem{bib02}
    Taro Hogeyama, Jiro Hogeyama:
    \newblock The Theory of Hoge,
    \newblock {\it The Proceedings of The Hoge Society}, 2008.
\end{thebibliography}


% 謝辞
%\include{src/acknowledgement.tex}

% 付録
\appendix

%\include{src/appendix.tex}

\end{document}
